\documentclass[a4,useAMS,usenatbib,usegraphicx,12pt]{article}
%External Packages and personalized macros
%=========================================================================
%		EXTERNAL PACKAGES
%=========================================================================
\usepackage[round]{natbib}
\usepackage[margin=3cm]{geometry}
\usepackage{hyperref}
\usepackage{times}
\usepackage{amsmath} 
\usepackage{amssymb}
\usepackage{fancyhdr,graphicx}
\usepackage{array, xcolor, bibentry}
\usepackage[utf8]{inputenc}
%\usepackage[nottoc, notlof, notlot]{tocbibind}

%\definecolor{lightgray}{gray}{0.8}
\newcolumntype{L}{>{\raggedleft}p{0.14\textwidth}}
\newcolumntype{R}{p{0.8\textwidth}}
\newcommand\VRule{\color{lightgray}\vrule width 0.5pt}

\usepackage{booktabs}% http://ctan.org/pkg/booktabs
\newcommand{\tabitem}{~~\llap{\textbullet}~~}

%=========================================================================
%		INTERNAL MACROS
%=========================================================================
% To highlight comments 
\definecolor{red}{rgb}{1,0.0,0.0}
\newcommand{\red}{\color{red}}
\definecolor{darkgreen}{rgb}{0.0,0.5,0.0}
\newcommand{\SRK}[1]{\textcolor{darkgreen}{\bf SRK: \textit{#1}}}
\newcommand{\SRKED}[1]{\textcolor{darkgreen}{\bf #1}}

\newcommand{\LCDM}{$\Lambda$CDM~}
\newcommand{\beq}{\begin{eqnarray}}  
\newcommand{\eeq}{\end{eqnarray}}  
\newcommand{\zz}{$z\sim 3$} 
\newcommand{\apj}{ApJ}  
\newcommand{\apjs}{ApJS}  
\newcommand{\apjl}{ApJL}  
\newcommand{\aj}{AJ}  
\newcommand{\mnras}{MNRAS}  
\newcommand{\mnrassub}{MNRAS accepted}  
\newcommand{\aap}{A\&A}  
\newcommand{\aaps}{A\&AS}  
\newcommand{\araa}{ARA\&A}  
\newcommand{\nat}{Nature}  
\newcommand{\physrep}{PhR}
\newcommand{\pasp}{PASP}    
\newcommand{\pasj}{PASJ}    
\newcommand{\avg}[1]{\langle{#1}\rangle}  
\newcommand{\ly}{{\ifmmode{{\rm Ly}\alpha}\else{Ly$\alpha$}\fi}}
\newcommand{\hMpc}{{\ifmmode{h^{-1}{\rm Mpc}}\else{$h^{-1}$Mpc }\fi}}  
\newcommand{\hGpc}{{\ifmmode{h^{-1}{\rm Gpc}}\else{$h^{-1}$Gpc }\fi}}  
\newcommand{\hmpc}{{\ifmmode{h^{-1}{\rm Mpc}}\else{$h^{-1}$Mpc }\fi}}  
\newcommand{\hkpc}{{\ifmmode{h^{-1}{\rm kpc}}\else{$h^{-1}$kpc }\fi}}  
\newcommand{\hMsun}{{\ifmmode{h^{-1}{\rm {M_{\odot}}}}\else{$h^{-1}{\rm{M_{\odot}}}$}\fi}}  
\newcommand{\hmsun}{{\ifmmode{h^{-1}{\rm {M_{\odot}}}}\else{$h^{-1}{\rm{M_{\odot}}}$}\fi}}  
\newcommand{\Msun}{{\ifmmode{{\rm {M_{\odot}}}}\else{${\rm{M_{\odot}}}$}\fi}}  
\newcommand{\msun}{{\ifmmode{{\rm {M_{\odot}}}}\else{${\rm{M_{\odot}}}$}\fi}}  
\newcommand{\lya}{{Lyman$\alpha$~}}
\newcommand{\clara}{{\texttt{CLARA}}~}
\newcommand{\rand}{{\ifmmode{{\mathcal{R}}}\else{${\mathcal{R}}$ }\fi}}  


%MY COMMANDS #############################################################
\newcommand{\sub}[1]{\mbox{\scriptsize{#1}}}
\newcommand{\dtot}[2]{ \frac{ d #1 }{d #2} }
\newcommand{\dpar}[2]{ \frac{ \partial #1 }{\partial #2} }
\newcommand{\pr}[1]{ \left( #1 \right) }
\newcommand{\corc}[1]{ \left[ #1 \right] }
\newcommand{\lla}[1]{ \left\{ #1 \right\} }
\newcommand{\bds}[1]{\boldsymbol{ #1 }}
\newcommand{\oiint}{\displaystyle\bigcirc\!\!\!\!\!\!\!\!\int\!\!\!\!\!\int}
\newcommand{\mathsize}[2]{\mbox{\fontsize{#1}{#1}\selectfont $#2$}}
\newcommand{\eq}[2]{\begin{equation} \label{eq:#1} #2 \end{equation}}
\newcommand{\lth}{$\lambda_{th}$ }
%#########################################################################

\setlength\parindent{0pt}

\def\LOGO{%
\begin{picture}(0,0)\unitlength=1cm
\put (6,-2) {\includegraphics[keepaspectratio=true,width=0.1\textheight]{./figures/UdeA_Shield.jpeg}}
\end{picture}
}

\begin{document}
\begin{flushleft}
  \sffamily\bfseries
  {\Large Computational Methods}\LOGO\\
  Universidad de Antioquia\\
  Facultad de Ciencias Exactas y Naturales\\
  Second Term 2014
\end{flushleft}

\hrulefill\par

\vspace{1cm}
%==============================================================================
\begin{tabular}{L!{\VRule}R}
\bf Professor		& Sebastian Bustamante\\
\bf E-mail			& sbustamante.academic \textit{at} gmail.com\\
\bf Classroom		& Computer room 6-125\\
\bf Class Time		& M-J 8.00-10.00 am \\
\bf Advisory		& --
\end{tabular}


%==============================================================================
\section*{Introduction}

This course is intended for students of Astronomy at the Universidad de Antioquia 
and will cover some numerical methods commonly used in science and specially in 
astronomy. These topics will be addressed from a formal context but also keeping 
a practical and computational approach, illustrating many useful applications in
problems of physics and astronomy.
%==============================================================================

%==============================================================================
\section*{Motivation}

As we understand more deeply our surrounding world, the involved phenomena
exhibit an ever increasing complexity and numerical solutions have become 
more and more common in physics and astronomy as an alternative (and a complement) 
to analytic solution. It is thus necessary to well understand the capabilities 
and limitations of these numerical methods as well as the type
of problems where they can be implemented.
%==============================================================================

%==============================================================================
\section*{Methodology}

Although a formal mathematical approach of the numerical methods is necessary, 
including deductions and error analysis, an \textit{algorithmic thinking} is more
than mandatory when we want to apply these methods to some specific problem. In
this context, \textit{algorithmic thinking} may be defined as the ability of 
abstract thinking in such a way we can tell the computer clearly how to do what 
we want it to do. Although a computer is not really required when applying 
numerical methods, the massive number of involved calculations makes more 
practical computational implementations.

\

As cycling, programming is a matter of practice, so in order to develop this 
\textit{algorithmic thinking} there is a strong practical component. This 
practical component will be almost entirely developed in \textit{Python} and 
slightly less in \textit{C} (when computational performance is required). 
However, students with knowledge in other programming languages (except
privative languages like MatLab, Mathematica) are also aimed to use them.

\

The course will be addressed in two steps each class: first, a formal 
introduction of the numerical methods will be presented, including deductions 
(when appropriate) and error analysis. For this part the students are expected 
to consult by their own before class the respective topic, so they can go 
quickly to the practical part. Secondly, the practical part will cover short
applications of the presented methods as well as some examples developed by
the professor during class.
%==============================================================================

%==============================================================================
\section*{Material}

All the material of the course will be in the next repository:

\url{ https://github.com/sbustamante/ComputationalMethods }

This program along with notes, presentations, solved examples, extra material
and homework can be found in there. The repository will be updated as the
course advances.
%==============================================================================

%==============================================================================
\section*{Evaluation}

The course is divided into three main blocks: the first includes numerical
methods for general algebra; the second block is related to numerical methods 
for calculus and linear algebra; finally the third block is about numerical 
methods for differential equations and statistics. Each one of these blocks 
will be evaluated with an exam of $25\%$. Additionally, weekly homework will 
cover the last $25\%$.

\

For the homework, some bonuses will be taken into account, for example the 
faster code, the more accurate, and so on. These bonuses would add $0.5$ to 
the grade of the respective task. The three exams will be held during class
time and may involved computational activities as well as analytical work.
%==============================================================================

\newpage
%==============================================================================
\section*{Program}

\subsection*{0. Overview of Python}
Basic commands of Python. Scripting. Scientific Libraries (Numpy, Scipy). 
Plotting with Matplotlib. IPython notebooks.

\subsection*{1. Mathematical preliminaries}
Round-off errors. Computer arithmetic. Algorithms and convergence.

\subsection*{2. One variable equations}
Transcendental equations. Bisection method. Fixed-Point iteration. Newton's 
Method. Error analysis. Zeros of polynomials. Applications.

\subsection*{3. Interpolation}
Linear interpolation. Lagrange polynomials. Divided differences. Hermite 
interpolation. Cubic spline interpolation. Applications.

\subsection*{4. Numerical calculus}
Numerical derivation. Numerical integration. Composite numerical integration.
Adaptive Quadrature Methods. Multiple integrals. Improper Integrals. 
Applications.

\subsection*{5. Linear algebra}
Linear systems of equations. Iterative techniques. Inverses and determinants.
LU factorization. Cholesky factorization. Pivoting strategies. Applications.

\subsection*{6. Differential equations}
First order methods: Euler, leap frog. Second order methods: Runge Kutta methods.
Systems of differential equations. Boundary and initial conditions. Applications.

\subsection*{7. Statistics}
Data adjust. Least square and non-linear least square. Random numbers. Monte 
Carlo techniques. Descriptive statistics. Applications.
%==============================================================================

\newpage
%==============================================================================
\section*{Schedule}

\begin{table}[h]
\begin{flushleft}
\begin{center}
  \begin{tabular}{l  l  l} \hline\hline
	\centering\textbf{Class} & \textbf{Date} & \textbf{Topics} \\ \hline
	0 & Oct 21& Presentation of the course. \\
	& & \textbf{0. Overview of Python} \\
	1 & Oct 23& Overview. Basic scripting. \\
	2 & Oct 30& Scientific libraries. Matplotlib. \\
	3 & Nov 04& IPython notebooks. \\
	& & \textbf{1. Mathematical Preliminaries} \\
	4 & Nov 06& Computer arithmetic. Round-off methods. \\
	5 & Nov 11& Algorithms and convergence. \\
	& & \textbf{2. One variable equations} \\
	6 & Nov 13& Bisection method. \\
	7 & Nov 18& Fixed-point iteration. \\
	8 & Nov 20& Newton's methods. Error analysis. \\
	9 & Nov 25& Zeros of polynomials. \\
	& & \textbf{3. Interpolation} \\
	10 & Nov 25& Linear interpolation. Lagrange polynomials. \\
	11 & Nov 27& Divided differences. Hermite interpolation. \\
	12 & Dic 02& Cubic spline interpolation. \\
	13 & Dic 09& \textbf{EXAM 1 (Topics 1, 2 and 3)} \\ \hline
	& & \textbf{Numerical calculus} \\
	\hline\hline
  \end{tabular}  
\end{center}
\end{flushleft}
\end{table}

%==============================================================================
\section*{Bibliography}
\begin{itemize}
\item \textit{Numerical Analysis}, Richard L. Burden \& J.Douglas Faires. Ninth edition. 2011.
\item \textit{Numerical Recipes, the Art of Scientific Computing}, William H. Press, Saul A. Teukolsky,
William T. Vetterling \& Brian P. Flannery. Third edition. The Cambridge University Press. 2007.
\item \textit{Introduction to Computational and Programming Using Python}, Guttag, J. V. 
The MIT Press. 2013.
\end{itemize}

%==============================================================================
\end{document}
