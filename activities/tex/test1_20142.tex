\documentclass[a4,useAMS,usenatbib,usegraphicx,12pt]{article}
%External Packages and personalized macros
%=========================================================================
%		EXTERNAL PACKAGES
%=========================================================================
\usepackage[round]{natbib}
\usepackage[margin=3cm]{geometry}
\usepackage{hyperref}
\usepackage{times}
\usepackage{amsmath} 
\usepackage{amssymb}
\usepackage{fancyhdr,graphicx}
\usepackage{array, xcolor, bibentry}
\usepackage[utf8]{inputenc}
%\usepackage[nottoc, notlof, notlot]{tocbibind}

%\definecolor{lightgray}{gray}{0.8}
\newcolumntype{L}{>{\raggedleft}p{0.14\textwidth}}
\newcolumntype{R}{p{0.8\textwidth}}
\newcommand\VRule{\color{lightgray}\vrule width 0.5pt}

\usepackage{booktabs}% http://ctan.org/pkg/booktabs
\newcommand{\tabitem}{~~\llap{\textbullet}~~}

%=========================================================================
%		INTERNAL MACROS
%=========================================================================
% To highlight comments 
\definecolor{red}{rgb}{1,0.0,0.0}
\newcommand{\red}{\color{red}}
\definecolor{darkgreen}{rgb}{0.0,0.5,0.0}
\newcommand{\SRK}[1]{\textcolor{darkgreen}{\bf SRK: \textit{#1}}}
\newcommand{\SRKED}[1]{\textcolor{darkgreen}{\bf #1}}

\newcommand{\LCDM}{$\Lambda$CDM~}
\newcommand{\beq}{\begin{eqnarray}}  
\newcommand{\eeq}{\end{eqnarray}}  
\newcommand{\zz}{$z\sim 3$} 
\newcommand{\apj}{ApJ}  
\newcommand{\apjs}{ApJS}  
\newcommand{\apjl}{ApJL}  
\newcommand{\aj}{AJ}  
\newcommand{\mnras}{MNRAS}  
\newcommand{\mnrassub}{MNRAS accepted}  
\newcommand{\aap}{A\&A}  
\newcommand{\aaps}{A\&AS}  
\newcommand{\araa}{ARA\&A}  
\newcommand{\nat}{Nature}  
\newcommand{\physrep}{PhR}
\newcommand{\pasp}{PASP}    
\newcommand{\pasj}{PASJ}    
\newcommand{\avg}[1]{\langle{#1}\rangle}  
\newcommand{\ly}{{\ifmmode{{\rm Ly}\alpha}\else{Ly$\alpha$}\fi}}
\newcommand{\hMpc}{{\ifmmode{h^{-1}{\rm Mpc}}\else{$h^{-1}$Mpc }\fi}}  
\newcommand{\hGpc}{{\ifmmode{h^{-1}{\rm Gpc}}\else{$h^{-1}$Gpc }\fi}}  
\newcommand{\hmpc}{{\ifmmode{h^{-1}{\rm Mpc}}\else{$h^{-1}$Mpc }\fi}}  
\newcommand{\hkpc}{{\ifmmode{h^{-1}{\rm kpc}}\else{$h^{-1}$kpc }\fi}}  
\newcommand{\hMsun}{{\ifmmode{h^{-1}{\rm {M_{\odot}}}}\else{$h^{-1}{\rm{M_{\odot}}}$}\fi}}  
\newcommand{\hmsun}{{\ifmmode{h^{-1}{\rm {M_{\odot}}}}\else{$h^{-1}{\rm{M_{\odot}}}$}\fi}}  
\newcommand{\Msun}{{\ifmmode{{\rm {M_{\odot}}}}\else{${\rm{M_{\odot}}}$}\fi}}  
\newcommand{\msun}{{\ifmmode{{\rm {M_{\odot}}}}\else{${\rm{M_{\odot}}}$}\fi}}  
\newcommand{\lya}{{Lyman$\alpha$~}}
\newcommand{\clara}{{\texttt{CLARA}}~}
\newcommand{\rand}{{\ifmmode{{\mathcal{R}}}\else{${\mathcal{R}}$ }\fi}}  


%MY COMMANDS #############################################################
\newcommand{\sub}[1]{\mbox{\scriptsize{#1}}}
\newcommand{\dtot}[2]{ \frac{ d #1 }{d #2} }
\newcommand{\dpar}[2]{ \frac{ \partial #1 }{\partial #2} }
\newcommand{\pr}[1]{ \left( #1 \right) }
\newcommand{\corc}[1]{ \left[ #1 \right] }
\newcommand{\lla}[1]{ \left\{ #1 \right\} }
\newcommand{\bds}[1]{\boldsymbol{ #1 }}
\newcommand{\oiint}{\displaystyle\bigcirc\!\!\!\!\!\!\!\!\int\!\!\!\!\!\int}
\newcommand{\mathsize}[2]{\mbox{\fontsize{#1}{#1}\selectfont $#2$}}
\newcommand{\eq}[2]{\begin{equation} \label{eq:#1} #2 \end{equation}}
\newcommand{\lth}{$\lambda_{th}$ }
%#########################################################################

\setlength\parindent{0pt}

\def\LOGO{%
\begin{picture}(0,0)\unitlength=1cm
\put (6,-2) {\includegraphics[keepaspectratio=true,width=0.1\textheight]{./figures/UdeA_Shield.jpeg}}
\end{picture}
}

\begin{document}
\begin{flushleft}
  \sffamily\bfseries
  {\Large Parcial 1\hspace{5cm}}\LOGO\\
  Métodos Computacionales \\
  Universidad de Antioquia\\
  Facultad de Ciencias Exactas y Naturales\\  
  2014-2\\
  
\hrulefill\par

  Nombre: \\
  Cédula: \\
  \footnotesize 22 Enero 2015

\hrulefill\par
  
  \end{flushleft}


%==============================================================================

El parcial tiene una duración de 2 horas. Cada numeral tiene un valor del $25\%$.

\begin{itemize}

 \item[\textbf{1.}] 
 El método de bisección para determinar raíces de funciones presenta una
 convergencia que escala como $1/2^k$, donde $k$ es el número de iteraciones 
 realizadas. Una modificación de este método consiste en dividir cada intervalo
 en $N$ subintervalos en vez de solamente dos. Este conjunto de métodos se pueden
 denominar métodos de $N$-sección. 
 
 Una ventaja de estos métodos es que la convergencia numérica mejora 
 sustancialmente ya que escala aproximadamente como $1/N^k$. Sin embargo, en 
 términos de rendimiento computacional esta ventaja se ve disminuida debido a que 
 una sola iteración de $N$-sección requiere más tiempo que una iteración de 
 bisección.
 
 Suponga que la tarea de dividir y evaluar el signo de un subintervalo toma un 
 tiempo $t_{int}$, que es el mismo para cualquiera de los métodos. \textbf{Calcule 
 para los métodos $N$-sección cuanto tiempo, en unidades de $t_{int}$, se requiere 
 para realizar $k$ iteraciones. Usando esta información y la formula para calcular
 la convergencia (precisión), considera usted que existe alguna ventaja en usar
 un método en específico?}
 
 \textbf{Pistas:} UNA iteración de bisección toma un tiempo de $2\ t_{int}$ ya 
 que es necesario dividir en dos subintevalosy evaluar el signo en cada uno de 
 ellos. 
 
 Si lo considera necesario, cree una tabla para bisección, trisección (3-sección) y 
 quadri\-sección (4-sección) donde tabule, en términos de cada iteración, la 
 convergencia y el tiempo acumulado requerido.
 
 
 \item[\textbf{2.}] 
 El método de Newton para encontrar raíces permite evaluar determinar la 
 solución a partir de un conjunto de iteraciones, donde la $n$-ésima aproximación
 está dada por
 
 \[ p_{n+1} = p_n - \frac{f(p_n)}{f'(p_n)} \]
 
 Calcular la raíz de la función
 
 \[f(x) = \cos(x)-x\]
 
 dando como valor inicial $p_n = 0$.
 
 \textbf{Tabule los resultados para cada iteración hasta que la aproximación coincida
 en 5 cifras decimales con la respuesta correcta ($0.73908513321516064$).}
 
 \item[\textbf{3.}] 
 Demostrar la formula general para los polinomios interpolantes de Lagrange
 
 \[P_n(x) = \sum_{i=0}^n y_i L_{n,i}(x)\]
 
 donde $n$ es el número de datos, $\{x_i\}_i$ y $\{y_i\}_i$ son los datos que se 
 desean interpolar y $L_{n,i}(x)$ son las funciones base.
 
 \textbf{Pista:} demuestre la expresión para $n=2$ puntos, y generalice la 
 expresión.

 \item[\textbf{4.}] 
 Usando los datos $X = \{0.5, 1.2, 3.2\}$ y $Y = \{-1.5, 0.1, 1.4\}$ y la 
 expresión deducida en el numeral anterior, determinar el polinomio interpolante 
 de Lagrange para estos puntos. 
 
 Escriba la respuesta de forma factorizada, es decir, $P_n(x) = a_0 + a_1 x + 
 a_2 x^2+\cdots$
 
\end{itemize}

%==============================================================================

\end{document}
