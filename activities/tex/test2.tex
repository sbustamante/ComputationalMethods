\documentclass[a4,useAMS,usenatbib,usegraphicx,12pt]{article}
%External Packages and personalized macros
%=========================================================================
%		EXTERNAL PACKAGES
%=========================================================================
\usepackage[round]{natbib}
\usepackage[margin=3cm]{geometry}
\usepackage{hyperref}
\usepackage{times}
\usepackage{amsmath} 
\usepackage{amssymb}
\usepackage{fancyhdr,graphicx}
\usepackage{array, xcolor, bibentry}
\usepackage[utf8]{inputenc}
%\usepackage[nottoc, notlof, notlot]{tocbibind}

%\definecolor{lightgray}{gray}{0.8}
\newcolumntype{L}{>{\raggedleft}p{0.14\textwidth}}
\newcolumntype{R}{p{0.8\textwidth}}
\newcommand\VRule{\color{lightgray}\vrule width 0.5pt}

\usepackage{booktabs}% http://ctan.org/pkg/booktabs
\newcommand{\tabitem}{~~\llap{\textbullet}~~}

%=========================================================================
%		INTERNAL MACROS
%=========================================================================
% To highlight comments 
\definecolor{red}{rgb}{1,0.0,0.0}
\newcommand{\red}{\color{red}}
\definecolor{darkgreen}{rgb}{0.0,0.5,0.0}
\newcommand{\SRK}[1]{\textcolor{darkgreen}{\bf SRK: \textit{#1}}}
\newcommand{\SRKED}[1]{\textcolor{darkgreen}{\bf #1}}

\newcommand{\LCDM}{$\Lambda$CDM~}
\newcommand{\beq}{\begin{eqnarray}}  
\newcommand{\eeq}{\end{eqnarray}}  
\newcommand{\zz}{$z\sim 3$} 
\newcommand{\apj}{ApJ}  
\newcommand{\apjs}{ApJS}  
\newcommand{\apjl}{ApJL}  
\newcommand{\aj}{AJ}  
\newcommand{\mnras}{MNRAS}  
\newcommand{\mnrassub}{MNRAS accepted}  
\newcommand{\aap}{A\&A}  
\newcommand{\aaps}{A\&AS}  
\newcommand{\araa}{ARA\&A}  
\newcommand{\nat}{Nature}  
\newcommand{\physrep}{PhR}
\newcommand{\pasp}{PASP}    
\newcommand{\pasj}{PASJ}    
\newcommand{\avg}[1]{\langle{#1}\rangle}  
\newcommand{\ly}{{\ifmmode{{\rm Ly}\alpha}\else{Ly$\alpha$}\fi}}
\newcommand{\hMpc}{{\ifmmode{h^{-1}{\rm Mpc}}\else{$h^{-1}$Mpc }\fi}}  
\newcommand{\hGpc}{{\ifmmode{h^{-1}{\rm Gpc}}\else{$h^{-1}$Gpc }\fi}}  
\newcommand{\hmpc}{{\ifmmode{h^{-1}{\rm Mpc}}\else{$h^{-1}$Mpc }\fi}}  
\newcommand{\hkpc}{{\ifmmode{h^{-1}{\rm kpc}}\else{$h^{-1}$kpc }\fi}}  
\newcommand{\hMsun}{{\ifmmode{h^{-1}{\rm {M_{\odot}}}}\else{$h^{-1}{\rm{M_{\odot}}}$}\fi}}  
\newcommand{\hmsun}{{\ifmmode{h^{-1}{\rm {M_{\odot}}}}\else{$h^{-1}{\rm{M_{\odot}}}$}\fi}}  
\newcommand{\Msun}{{\ifmmode{{\rm {M_{\odot}}}}\else{${\rm{M_{\odot}}}$}\fi}}  
\newcommand{\msun}{{\ifmmode{{\rm {M_{\odot}}}}\else{${\rm{M_{\odot}}}$}\fi}}  
\newcommand{\lya}{{Lyman$\alpha$~}}
\newcommand{\clara}{{\texttt{CLARA}}~}
\newcommand{\rand}{{\ifmmode{{\mathcal{R}}}\else{${\mathcal{R}}$ }\fi}}  


%MY COMMANDS #############################################################
\newcommand{\sub}[1]{\mbox{\scriptsize{#1}}}
\newcommand{\dtot}[2]{ \frac{ d #1 }{d #2} }
\newcommand{\dpar}[2]{ \frac{ \partial #1 }{\partial #2} }
\newcommand{\pr}[1]{ \left( #1 \right) }
\newcommand{\corc}[1]{ \left[ #1 \right] }
\newcommand{\lla}[1]{ \left\{ #1 \right\} }
\newcommand{\bds}[1]{\boldsymbol{ #1 }}
\newcommand{\oiint}{\displaystyle\bigcirc\!\!\!\!\!\!\!\!\int\!\!\!\!\!\int}
\newcommand{\mathsize}[2]{\mbox{\fontsize{#1}{#1}\selectfont $#2$}}
\newcommand{\eq}[2]{\begin{equation} \label{eq:#1} #2 \end{equation}}
\newcommand{\lth}{$\lambda_{th}$ }
%#########################################################################

\setlength\parindent{0pt}

\def\LOGO{%
\begin{picture}(0,0)\unitlength=1cm
\put (6,-2) {\includegraphics[keepaspectratio=true,width=0.1\textheight]{./figures/UdeA_Shield.jpeg}}
\end{picture}
}

\begin{document}
\begin{flushleft}
  \sffamily\bfseries
  {\Large Parcial 2\hspace{5cm}}\LOGO\\
  Métodos Computacionales \\
  Universidad de Antioquia\\
  Facultad de Ciencias Exactas y Naturales\\  
  2014-2\\
  
\hrulefill\par

  Nombre: \\
  Cédula: \\
  \footnotesize 26 Febrero 2015

\hrulefill\par
  
  \end{flushleft}


%==============================================================================

El parcial tiene una duración de 2 horas. Cada numeral tiene un valor del $33.33\%$.

\begin{itemize}

 \item[\textbf{1.}] 
Para derivar numéricamente una función $f(x)$ se pueden usar varias aproximaciones.
La primera y más trivial está dada por:

$$ f'(x) = \frac{f(x+h)-f(x)}{h} $$

para un paso $h$ que sea suficientemente pequeño.

Aproximaciones mejores pueden ser derivadas a partir la fórmula de $n$-puntos 
vista durante clase. Como casos especiales se tiene la fórmula de punto medio para
3 y 5 puntos, dadas respectivamente por:

$$ f'(x) = \frac{1}{2h} [ f(x+h)-f(x-h) ] $$

$$ f'(x) = \frac{1}{12h} [ f(x-2h)-8f(x-h)+8f(x+h)-f(x+2h) ] $$

Tome la función $f(x) = x^2 \cos(x)$ y derívela analíticamente en $x=2$. Calcule la
misma derivada usando las tres anteriores aproximaciones para valores de $h=0.5, 0.1, 
0.05, 0.01$. Realice una tabla y tabule el error relativo para cada aproximación
y para cada paso $h$. ¿Qué puede concluir del comportamiento del error con respecto
al paso $h$ para cada aproximación?

  \item[\textbf{2.}] Asuma una función $f(x)$ definida sobre un intervalo $[a,b]$.
Usando el polinomio interpolante de Lagrange de orden $n$, es posible aproximar la
integral de la función sobre el intervalo $[a,b]$ a través de la siguiente 
expresión:

$$ \int_a^b f(x) dx \approx \sum_{i=0}^n a_i f(x_i)  $$

donde

$$ a_i = \int_a^b \prod_{j=0,j\neq i}^n \frac{(x-x_j)}{(x_i-x_j)}dx $$.

A partir de esto, demuestre la regla de trapecio:

$$ \int_a^b f(x)dx= \frac{h}{2}[ f(a)+f(b) ] $$

con $h=b-a$.

Y la regla de Simpson:

$$ \int_a^b f(x)dx= \frac{h}{3}[ f(x_0)+4f(x_1)+f(x_2) ] $$

con $x_0=a$, $x_1 = (a+b)/2$, $x_2 =b$ y $h = (b-a)/2$.
 
 
 \item[\textbf{3.}] Dada una matriz no singular $A$ de tamaño $n\times n$, es posible
 calcular su determinante a partir de las siguientes definiciones:
 
 \begin{itemize}
  \item Si $A=[a]$ es una matriz de tamaño $1\times 1$, el determinante es $\det(A) = a$.
  \item Si $A$ es una matriz $2\times 2$ dada por:
  
  $$ A = \left[\begin{matrix} a_{11} & a_{12} \\ a_{21} & a_{22} \end{matrix}\right] $$
  
  su determinante está dado por
  
  $$ \det(A) = \left|\begin{matrix} a_{11} & a_{12} \\ a_{21} & a_{22} \end{matrix}\right| = a_{11}a_{22}-a_{12}a_{21} $$
  
  \item Si $A$ es una matriz $n\times n$, el menor $M_{ij}$ está definido como el determinante de la matrix $(n-1)\times (n-1)$
  obtenida a partir de eliminar la $i$-ésima fila y la $j$-ésima columna de $A$.
  
  \item El cofactor $A_{ij}$ asociado al menor $M_{ij}$ está definido por:
  
  $$ A_{ij} = (-1)^{i+j}M_{ij} $$.
  
  \item El determinante de la matriz $A$ puede calcularse a partir de cualquiera de las siguientes
  expresiones:
  
  $$ \det(A) = \sum_{j=1}^n a_{ij}A_{ij} $$
  
  ó
  
  $$ \det(A) = \sum_{i=1}^n a_{ij}A_{ij} $$
  
  Es decir, puede usarse una columna o una fila para la sumatoria.
  
 \end{itemize}

A partir de estas propiedades, cualquier determinante puede ser calculado de forma recursiva.
Demueste que el número de multiplicaciones que son requeridas para calcular el determinante 
de una matriz $A$ de tamaño $n\times n$, cuando $n$ es grande, está dado por:

$$ N_{mult} \approx n!e $$

donde $e$ es el número de Euler.

Teniendo en cuenta que para un computador toma un tiempo mayor realizar una operación de 
multiplicación/división que una operación suma/resta, $N_{mult}$ representa también el tiempo
de cómputo total del algorítmo en unidades del tiempo inidividual de cada operación.
 
 
\textbf{Ayuda:} el término $(-1)^{i+j}$ del cofactor $A_{ij}$ no representa un tiempo considerable
para un computador puesto que el valor será siempre $1$ o $-1$ y la multiplicación es trivial.
Por lo tanto, no considere esta multiplicación cuando calcule $N_{mult}$.
 
\end{itemize}

%==============================================================================

\end{document}
